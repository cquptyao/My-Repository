%
% ---------------------------------------------------------------
% Copyright (C) 2012-2018 Gang Li
% ---------------------------------------------------------------
%
% This work is the default powerdot-tuliplab style test file and may be
% distributed and/or modified under the conditions of the LaTeX Project Public
% License, either version 1.3 of this license or (at your option) any later
% version. The latest version of this license is in
% http://www.latex-project.org/lppl.txt and version 1.3 or later is part of all
% distributions of LaTeX version 2003/12/01 or later.
%
% This work has the LPPL maintenance status "maintained".
%
% This Current Maintainer of this work is Gang Li.
%
%

\documentclass[
size=14pt,
paper=smartboard,  %a4paper, smartboard, screen
mode=present, 		%present, handout, print
display=slides, 	% slidesnotes, notes, slides
style=tuliplab,  	% TULIP Lab style
pauseslide,
fleqn,leqno]{powerdot}


\usepackage{cancel}
\usepackage{caption}
\usepackage{stackengine}
\usepackage{smartdiagram}
\usepackage{attrib}
\usepackage{amssymb}
\usepackage{amsmath} 
\usepackage{amsthm} 
\usepackage{mathtools}
\usepackage{rotating}
\usepackage{graphicx}
\usepackage{boxedminipage}
\usepackage{rotate}
\usepackage{calc}
\usepackage[absolute]{textpos}
\usepackage{psfrag,overpic}
\usepackage{fouriernc}
\usepackage{pstricks,pst-3d,pst-grad,pstricks-add,pst-text,pst-node,pst-tree}
\usepackage{moreverb,epsfig,subfigure}
\usepackage{color}
\usepackage{booktabs}
\usepackage{etex}
\usepackage{breqn}
\usepackage{multirow}
\usepackage{natbib}
\usepackage{bibentry}
\usepackage{gitinfo2}
\usepackage{siunitx}
\usepackage{nicefrac}
%\usepackage{geometry}
%\geometry{verbose,letterpaper}
\usepackage{media9}
\usepackage{animate}
%\usepackage{movie15}
\usepackage{auto-pst-pdf}

\usepackage{breakurl}
\usepackage{fontawesome}
\usepackage{xcolor}
\usepackage{multicol}



\usepackage{verbatim}
\usepackage[utf8]{inputenc}
\usepackage{dtk-logos}
\usepackage{tikz}
\usepackage{adigraph}
%\usepackage{tkz-graph}
\usepackage{hyperref}
%\usepackage{ulem}
\usepackage{pgfplots}
\usepackage{verbatim}
\usepackage{fontawesome} 


\usepackage{todonotes}
% \usepackage{pst-rel-points}
\usepackage{animate}
\usepackage{fontawesome}
\usepackage{graphicx}
\usepackage{listings}
\lstset{frameround=fttt,
	frame=trBL,
	stringstyle=\ttfamily,
	backgroundcolor=\color{yellow!20},
	basicstyle=\footnotesize\ttfamily}
\lstnewenvironment{code}{
	\lstset{frame=single,escapeinside=`',
		backgroundcolor=\color{yellow!20},
		basicstyle=\footnotesize\ttfamily}
}{}


\usepackage{hyperref}
\hypersetup{ % TODO: PDF meta Data
	pdftitle={Presentation Title},
	pdfauthor={Gang Li},
	pdfpagemode={FullScreen},
	pdfborder={0 0 0}
}


% \usepackage{auto-pst-pdf}
% package to show source code

\definecolor{LightGray}{rgb}{0.9,0.9,0.9}
\newlength{\pixel}\setlength\pixel{0.000714285714\slidewidth}
\setlength{\TPHorizModule}{\slidewidth}
\setlength{\TPVertModule}{\slideheight}
\newcommand\highlight[1]{\fbox{#1}}
\newcommand\icite[1]{{\footnotesize [#1]}}

\newcommand\twotonebox[2]{\fcolorbox{pdcolor2}{pdcolor2}
	{#1\vphantom{#2}}\fcolorbox{pdcolor2}{white}{#2\vphantom{#1}}}
\newcommand\twotoneboxo[2]{\fcolorbox{pdcolor2}{pdcolor2}
	{#1}\fcolorbox{pdcolor2}{white}{#2}}
\newcommand\vpspace[1]{\vphantom{\vspace{#1}}}
\newcommand\hpspace[1]{\hphantom{\hspace{#1}}}
\newcommand\COMMENT[1]{}

\newcommand\placepos[3]{\hbox to\z@{\kern#1
		\raisebox{-#2}[\z@][\z@]{#3}\hss}\ignorespaces}

\renewcommand{\baselinestretch}{1.2}


\newcommand{\draftnote}[3]{
	\todo[author=#2,color=#1!30,size=\footnotesize]{\textsf{#3}}	}
% TODO: add yourself here:
%
\newcommand{\gangli}[1]{\draftnote{blue}{GLi:}{#1}}
\newcommand{\shaoni}[1]{\draftnote{green}{sn:}{#1}}
\newcommand{\gliMarker}
{\todo[author=GLi,size=\tiny,inline,color=blue!40]
	{Gang Li has worked up to here.}}
\newcommand{\snMarker}
{\todo[author=Sn,size=\tiny,inline,color=green!40]
	{Shaoni has worked up to here.}}

%%%%%%%%%%%%%%%%%%%%%%%%%%%%%%%%%%%%%%%%%%%%%%%%%%%%%%%%%%%%%%%%%%%%%%%%
% title
% TODO: Customize to your Own Title, Name, Address
%
\title{Bike Sharing Demand}
\author{
	Yao Yang
	\\
	\\Chongqing University of Posts and Telecommunications
}
\date{\today}


% Customize the setting of slides
\pdsetup{
	% TODO: Customize the left footer, and right footer
	rf=\href{http://www.tulip.org.au}{
		Last Changed by: \textsc{\gitCommitterName}\ \gitVtagn-\gitAbbrevHash\ (\gitAuthorDate)
	},
	cf={Group Outlying Aspects Mining},
}


\begin{document}
	
	\maketitle
	
	%\begin{slide}{Overview}
	%\tableofcontents[content=sections]
	%\end{slide}
	
	
	%%==========================================================================================
	%%
	\begin{slide}[toc=,bm=]{Overview}
		\tableofcontents[content=currentsection,type=1]
	\end{slide}
	%%
	%%==========================================================================================
	
	
	\section{Problem Definition}
	
	
	%%==========================================================================================
	%%
	\begin{slide}{Bike Sharing Demand}
		
		\begin{center}
			\twotonebox{\rotatebox{90}{Defn}}{\parbox{.86\textwidth}
				{The goal of this project is to forecast bike rental demand given the input feature like the duration of travel, departure location, arrival location, and time elapsed.}}
			
			
			\twotonebox{\rotatebox{90}{Defn}}{\parbox{.86\textwidth}
				{Evaluation metrics: RMSLE(Root Mean Squard Logarithmic Error) is required to evaluate the model.
					\\$ RMSLE =  \sqrt{\tfrac{1}{n}\sum_{i=1}^n\left[log(p_i+1)-log(a_i+1) \right]^2} $
					\\n is the number of test set samples, pi is the test value, and ai is the actual value. When the root mean square error is smaller, it means that the fitting effect of the data is better and the test value is closer to the actual value.
				}
			}
		\end{center}
		%%==========================================================================================
	\end{slide}
	%%==========================================================================================
	
	
	
	\section{Data clean}
	%%==========================================================================================
	%%
	\begin{slide}{Date describe}
		\begin{center}
			\twotonebox{\rotatebox{90}{Defn}}{\parbox{.86\textwidth}
				{You are provided hourly rental data spanning two years. For this competition, the training set is comprised of the first 19 days of each month, while the test set is the 20th to the end of the month. You must predict the total count of bikes rented during each hour covered by the test set, using only information available prior to the rental period.
					\begin{itemize}
						\item \textcolor{orange}{train.csv} It contains a training set of target variables.
						\item \textcolor{orange}{test.csv} It does not contain a training set of target variables.
						\item \textcolor{orange}{sampleSubmission.csv} It is a properly formatted sample submission file.
					\end{itemize}
			}}
		\end{center}
		
	\end{slide}
	%%==========================================================================================
	
	%%==========================================================================================
	%%
	\begin{slide}{Date fields}
		
		\begin{itemize}
			\item \textcolor{orange}{datetime} - hourly date + timestamp
			\item \textcolor{orange}{season} - 1 = spring, 2 = summer, 3 = fall, 4 = winter 
			\item \textcolor{orange}{holiday} - whether the day is considered a holiday
			\item \textcolor{orange}{workingday}  - whether the day is neither a weekend nor holiday
			\item \textcolor{orange}{weather} - 1: Clear, Few clouds, Partly cloudy, Partly cloudy
			\\2: Mist + Cloudy, Mist + Broken clouds, Mist + Few clouds, Mist
			\\3: Light Snow, Light Rain + Thunderstorm + Scattered clouds, Light Rain + Scattered clouds
			\\4: Heavy Rain + Ice Pallets + Thunderstorm + Mist, Snow + Fog
			\item \textcolor{orange}{temp} - temperature in Celsius
			\item \textcolor{orange}{atemp} - "feels like" temperature in Celsius
			\item \textcolor{orange}{humidity} - relative humidity
			\item \textcolor{orange}{windspeed} - wind speed
			\item \textcolor{orange}{casual} - number of non-registered user rentals initiated
			\item \textcolor{orange}{registered} - number of registered user rentals initiated
			\item \textcolor{orange}{count} - number of total rentals
		\end{itemize}
		
	\end{slide}
	%%==========================================================================================
	
	\section{Implementation process}
	
	
	%%==========================================================================================
	%%
	\begin{slide}{Step One - Data preprocessing}
		%Step One - Group Feature Extraction}
	\begin{itemize}
		\item
		\smallskip
		Suppose $f_1$, $f_2$, $f_3$ are three features of $G_q$.
		
		$f_1$: \{$x_1, x_2, x_3, x_4, x_5, x_2, x_3, x_4, x_1, x_2$\} \\
		
		$f_2$: \{$y_2, y_2, y_1, y_2, y_3, y_3, y_5, y_4, y_4, y_2$\} \\
		
		$f_3$: \{$z_1, z_4, z_2, z_4, z_5, z_3, z_1, z_2, z_4, z_2$\} \\
	\end{itemize}
	
	\begin{figure}[htbp]
		\centering
		\subfigure[$f_1$]{
			\selectcolormodel{rgb}
			\missingfigure[figwidth=5.5cm]{Test.}
			%\includegraphics[width=0.25\textwidth]{figures//frequency-distribution-feature1.eps}
			\label{fig:fre-dis-f1}
		}
		\subfigure[$f_2$]{
			\selectcolormodel{rgb}
			\missingfigure[figwidth=5.5cm]{Test.}
			\label{fig:fre-dis-f2}
		}
		\subfigure[$f_3$]{
			\selectcolormodel{rgb}
			\missingfigure[figwidth=5.5cm]{Test.}
			\label{fig:fre-dis-f3}
		}
		\caption{Histogram of $G_q$ on three features}
		\label{fig:fre-dis-each-feature}
	\end{figure}
	
\end{slide}
%%
%%==========================================================================================


%%==========================================================================================
%%
\begin{slide}{Step Two - Feature engineering}
	%Step Two - Outlying Degree Scoring
	\begin{itemize}
		\item
		Calculate Earth Mover Distance
		
		\begin{itemize}
			\item
			Represent one feature among different groups
			
			\item
			Purpose: calculate the minimum mean distance
		\end{itemize}
		
		\begin{figure}
			\selectcolormodel{rgb}
			\missingfigure{Make a sketch of the structure of a trebuchet.}
			%  \includegraphics[width=0.4\textwidth]{figures//example3.eps}\\
			\caption{EMD of one feature}\label{EMD}
		\end{figure}
	\end{itemize}
	
	%%==========================================================================================
	\begin{note}
		The second step is outlying degree scoring,
		which is to evaluate the outlying degree between the target group and competitive groups.
		
		First,
		we calculate the earth mover distance of one feature in different groups.
		
		The earth mover distance reflects the minimum mean distance between
		the target group and other groups on one feature.
		
		Later on,
		we utilize the EMD to measure the differences between groups.
	\end{note}
	%%==========================================================================================
	
\end{slide}
%%
%%==========================================================================================


%%==========================================================================================
%%
\begin{slide}{Step Three - Buliding models to make predictions}
	
	\begin{itemize}
		\item
		Calculate the outlying degree
		
		\vspace{1.2cm}
		
		\begin{centering}
			
			$ OD(G_q) = \sum_{1}^{n}EDM(h_{q_s}, h_{k_s}) $
			
		\end{centering}
		
		\begin{itemize}
			\item
			n $\Leftrightarrow$ the number of contrast groups.
			
			\item
			$h_{k_s}$  $\Leftrightarrow$ the histogram representation of $G_k$ in the subspace s.
			
		\end{itemize}
	\end{itemize}
	
\end{slide}
%%
%%==========================================================================================


%%==========================================================================================
%%
\begin{slide}{Step Four - Selecting 4 optimal models for Stacking fusion.}
	%Step Three - Outlying Aspects Identification
	\begin{itemize}
		\item
		Identify group outlying aspects mining based on the value
		of outlying degree.
		
		\item
		The greater the outlying degree is,
		the more likely it is group outlying aspect.
	\end{itemize}
	
\end{slide}
%%
%%==========================================================================================

\section{Prediction results}
%%==========================================================================================
%%
\begin{slide}[toc=,bm=]{Evaluation}
	
	\begin{center}
		\begin{itemize}
			
			\item
			\smallskip
			\large
			{$Accuracy = \frac{P}{T}$ \\
				P: Identified outlying aspects \\
				
				T: Real outlying aspects}
			
		\end{itemize}
	\end{center}
	
	%%==========================================================================================
	\begin{note}
		Before showing the experiment results,
		I will introduce the evaluation of the experiment.
		
		We use accuracy formula to make comparisons between GOAM algorithm
		and outlying aspect mining method.
		In the accuracy formula,
		P stands for identified outlying aspects;
		and T means the real outlying aspects.
	\end{note}
	%%==========================================================================================
	
\end{slide}
%%
%%==========================================================================================


%%==========================================================================================
%%
\begin{slide}{Synthetic Dataset}
	
	\begin{itemize}
		\item Synthetic Dataset and Ground Truth
	\end{itemize}
	
	\begin{table}
		\setlength{\abovecaptionskip}{0pt}
		\setlength{\belowcaptionskip}{10pt}
		\centering
		\caption{Synthetic Dataset and Ground Truth}
		
		\begin{tabular}{p{2.8cm}p{0.9cm}p{0.9cm}p{0.9cm}p{0.9cm}p{0.9cm}p{0.9cm}p{0.9cm}p{0.9cm}}
			\hline
			% after \\: \hline or \cline{col1-col2} \cline{col3-col4} ...
			Query group  & $\mathbf{F_1}$ & $\mathbf{F_2}$ & $F_3$ & $\mathbf{F_4}$ & $F_5$ & $F_6$ & $F_7$ & $F_8$\\
			\hline
			$i_1$   & \bf{10} & \bf{8}  & 9  & \bf{7}  & 7 & 6 & 6  & 8\\
			$i_2$   & \bf{9}  & \bf{9}  & 7  & \bf{8}  & 9 & 9 & 8  & 9\\
			$i_3$   & \bf{8}  & \bf{10} & 8  & \bf{9}  & 6 & 8 & 7  & 8\\
			$i_4$   & \bf{8}  & \bf{8}  & 6  & \bf{7}  & 8 & 8 & 6  & 7\\
			$i_5$   & \bf{9}  & \bf{9}  & 9  & \bf{7}  & 7 & 7 & 8  & 8\\
			$i_6$   & \bf{8}  & \bf{10} & 8  & \bf{8}  & 6 & 6 & 8  & 7\\
			$i_7$   & \bf{9}  & \bf{9}  & 7  & \bf{9}  & 8 & 8 & 8  & 7\\
			$i_8$   & \bf{10} & \bf{9}  & 10 & \bf{7}  & 7 & 7 & 7  & 7\\
			$i_9$   & \bf{9}  & \bf{10} & 8  & \bf{8}  & 7 & 6 & 7  & 7\\
			$i_{10}$& \bf{9}  & \bf{9}  & 7  & \bf{7}  & 7 & 8 & 8  & 8\\
			\hline
		\end{tabular}
	\end{table}
	
	
\end{slide}
%%
%%==========================================================================================
\end{document}